Requirements for other parts of the thesis work can be found on the school
web-pages~\cite{Noo05}.  The requirements below are for the written thesis
only.

\section{Format}
The following format specifications must be adhered to for your thesis
(the \LaTeX\ template available from the school ensures this):
\begin{enumerate}
\item The thesis must be printed on \emph{A4 size paper}.
\item The thesis must be typed or prepared using a \emph{word-processor}.
You are encouraged to use both sides of the paper.
\item \emph{Margins} on all sides must be no less than \unit[25]{mm} (before
binding).
\item \emph{1.5 line spacing} (about \unit[8]{mm} per line) must be used.
\item All sheets must be \emph{numbered}. The main body of the thesis must be
numbered consecutively from beginning to end.  Other sections must either
be included or have their own logical numbering system.
\item The \emph{title page} must contain the following information:
\begin{enumerate}
\item University and School names.
\item Title of Thesis/Project.
\item Topic Number (if applicable).
\item Name of Author and student ID.
\item The degree the thesis is submitted for.
\item Submission date (month and year).
\item Supervisor's name.
\end{enumerate}
\end{enumerate}

\section{Other physical appearance}
Other requirements to the physical appearance of your theses are:
\begin{enumerate}
\item The report must be \emph{spiral bound} (at your own cost).
\item Formulas and other items difficult to type may be \emph{neatly
hand-written}
in \emph{permanent} black ink.
\item \emph{Graphs, diagrams and photographs} should be inserted as close as
possible to their \emph{first reference} in the text. Rotated
graphs etc are to be arranged so as to be conveniently read, with the
bottom edge to the outside of the page.
\emph{Graphs and diagrams must be legible!}
\item \emph{Photographs} must be permanently attached to sheets at least along
their left edge. Double sided adhesive may be
used to attach photographs. Photographs printed on A4 size lightweight
paper may be bound directly into the thesis.
\item \emph{Computer programs} and \emph{engineering drawings} should be bound into the
thesis, usually in an appendix.
\item \emph{Floppy diskettes/CD} may be attached to the back cover of the thesis
folder using self adhesive tape or in a secure
pocket.
\end{enumerate}

\section{Submission}

Finally, here are some requirements to the submission procedure. 

\begin{enumerate}
\item The \emph{author} of the thesis is \emph{responsible} for the preparation of the
thesis before the deadline, proofreading the
typescript and having corrections made as necessary.
\item All students must submit a \emph{thesis summary sheet} with their thesis
report. This summary sheet is designed to assist
in determining the overall input by students into the thesis work. Please
note that a separate summary sheet must
be submitted by individual student, even if part of a group submitting a
group thesis.
The guidelines for completing
the summary sheet and the summary sheet form can be downloaded from the
School Office Website.
\item \emph{Two copies} of each thesis/group thesis report must be submitted.
\item Students doing a \emph{Group Thesis} are required to write and hand in
\emph{individual reports}.  The reports should be
clearly distinguishable, and appropriately cross referenced to each other.
The common work overlapping between the reports should be clearly
identified.
\item There is a \emph{page limit} of 100 pages for the main body of the thesis.
\end{enumerate}



\chapter{Content Requirements}\label{ch:content}

Students should consult the literature (e.g.~\cite{Sid99,StrWhi79,Coo64})
and other resources for material on how to write a good
thesis.  The present document is only a very brief introduction as to what
is expected.

\nocite{NieLeh03,HasLehKwo05}

\section{Structure}
Most theses are structured very much like the present document.
The main part of the thesis can be structured in many different ways,
however, but must contain: a \emph{problem definition};
\emph{theory} and \emph{considerations} on how to solve the problem;
a description of the \emph{solution method} (dimensioning, construction,
etc.);
presentation of \emph{results} (measurements, simulations, etc.);
a \emph{discussion} of the results (validity, deviations, comparison
with previous solutions, etc.); and finally the \emph{conclusions}.

\section{Style of writing}

\begin{enumerate}

\item Audience:
The thesis must be addressed to engineers at the same level as the
student but without the special knowledge gained during the thesis work.
Such a third-person must be able to reconstruct the results on the basis
of the thesis alone.

\item
Every used concept/symbol/abbreviation which is not widely know must be \emph{defined}.
The wording should be \emph{short} and \emph{concise};  a suitable length
is 40--70 pages (plus appendices).
Readable(!) \emph{figures} and \emph{graphs} enhances comprehensibility.

\item Units.
\emph{SI units} must be used.
\end{enumerate}

\section{Documentation}

\begin{enumerate}
\item
The work must be well documented; i.e. enclosed must be the \emph{complete
schematics} of designed electronic circuits/test set-ups and/or a
\emph{program listing}, and/or etc.
Documentation of \emph{simulation results} and/or \emph{measurement
results} likewise.
\item References:
For every declaration/equation/method/etc., which is not widely known,
a \emph{reference to the literature} must be given (or a `proof' if it is
the authors own work).
In case material is copied verbatim, quotes must be used.
This is also the case when referring to partners
work in the case of a Group Thesis.

\item Plagiarism:
Failure to give proper references to the literature is \emph{plagiarism}.
Plagiarism is considered serious offence and severe penalties may apply.

\end{enumerate}

